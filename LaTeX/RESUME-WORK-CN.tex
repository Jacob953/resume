\documentclass[letterpaper,10pt]{article}

\usepackage{latexsym}
\usepackage[empty]{fullpage}
\usepackage{titlesec}
\usepackage{marvosym}
\usepackage[usenames,dvipsnames]{color}
\usepackage{verbatim}
\usepackage{enumitem}
\usepackage[hidelinks]{hyperref}
\usepackage{fancyhdr}
\usepackage[english]{babel}
\usepackage[UTF8]{ctex}

\pagestyle{fancy}
\fancyhf{} % clear all header and footer fields
\fancyfoot{}
\renewcommand{\headrulewidth}{0pt}
\renewcommand{\footrulewidth}{0pt}

% Adjust margins
\addtolength{\oddsidemargin}{-0.6in}
\addtolength{\evensidemargin}{-0.5in}
\addtolength{\textwidth}{1.2in}
\addtolength{\topmargin}{-.6in}
\addtolength{\textheight}{1.0in}

\urlstyle{same}

\raggedbottom
\raggedright
\setlength{\tabcolsep}{0in}

% Sections formatting
\titleformat{\section}{
  \vspace{-8pt}\scshape\raggedright\large
}{}{0em}{}[\color{black}\titlerule \vspace{-6pt}]

%-------------------------
% Custom commands
\newcommand{\resumeItemSingle}[1]{
  \item\small{
    {#1 \vspace{-2pt}}
  }
}
\newcommand{\resumeItem}[2]{
  \item\small{
    \textbf{\:#1}{ #2 \vspace{-4pt}}
  }
}

\newcommand{\resumeSubheadingMininal}[1]{
  \vspace{-0pt}\item
    \begin{tabular*}{0.97\textwidth}[t]{l@{\extracolsep{\fill}}r}
      \textbf{#1}
    \end{tabular*}\vspace{-6pt}
}

\newcommand{\resumeSubheadingCompact}[2]{
\vspace{-0pt}\item
    \begin{tabular*}{0.97\textwidth}[t]{l@{\extracolsep{\fill}}r}
      \textbf{#1} & \textit{\small #2}
    \end{tabular*}\vspace{-1pt}
    \vspace{-5pt}
}

\newcommand{\resumeSubheadingCompactVertical}[2]{
	\vspace{-0pt}\item
    \begin{tabular*}{0.97\textwidth}[t]{l@{\extracolsep{\fill}}r}
      \textbf{#1}
    \end{tabular*}\vspace{-0pt}
    #2
    \vspace{-6pt}
}

\newcommand{\resumeSubheadingExtended}[3]{
  \vspace{-0pt}\item
    \begin{tabular*}{0.97\textwidth}[t]{l@{\extracolsep{\fill}}r}
      \textbf{#1} & #2
    \end{tabular*}\vspace{-0pt}
    #3
    \vspace{-6pt}
}

\newcommand{\resumeSubheadingWithTime}[3]{
  \vspace{-0pt}\item
    \begin{tabular*}{0.97\textwidth}[t]{l@{\extracolsep{\fill}}r}
      \textbf{#1} & \textit{\small #2} \\
    \end{tabular*}\vspace{-0pt}
    #3
    \vspace{-6pt}
}

\newcommand{\resumeSubheading}[4]{
  \vspace{-1pt}\item
    \begin{tabular*}{0.97\textwidth}[t]{l@{\extracolsep{\fill}}r}
      \textbf{#1} & #2 \\
      \textit{\small#3} & \textit{\small #4} \\
    \end{tabular*}\vspace{-6pt}
}

\newcommand{\resumeSubItem}[2]{
    \vspace{-1pt}\item
    \begin{tabular*}{0.97\textwidth}[t]{l@{\extracolsep{\fill}}r}
        \textbf{#1} & \textit{\samll #2}
    \end{tabular*}\vspace{-14pt}
    }

\renewcommand{\labelitemii}{$\circ$}

\newcommand{\resumeSubHeadingListStart}{\begin{itemize}[leftmargin=*]\vspace{-1pt}}
\newcommand{\resumeSubHeadingListEnd}{\end{itemize}\vspace{-2pt}}
\newcommand{\resumeItemListStart}{\begin{itemize}}
\newcommand{\resumeItemListEnd}{\end{itemize}\vspace{-5pt}}


%-------------------------------------------
%%%%%%  CV STARTS HERE  %%%%%%%%%%%%%%%%%%%%%%%%%%%%


\begin{document}

%----------HEADING-----------------
\begin{tabular*}{\textwidth}{l@{\extracolsep{\fill}}r}
  \textbf{\Large{余俊锋}} \\
  \centerline{
  \footnotesize \textbf{Portf:} \href{https://jacob953.com/}{https://jacob953.com/} |
  \textbf{Email:} \href{mailto:i@jacob953.com}{i@jacob953.com} | 
  \textbf{LinkedIn:} \href{https://www.linkedin.cn/incareer/in/junfeng-yu-578456248}{https://www.linkedin.com/in/jacob953/} |
  \textbf{Mobile:} +86 155 8521 0953}
\end{tabular*}
\vspace{-10pt}
\vspace{-4pt}
%-----------Education Background-----------------
\vspace{-1pt}
\section{教育背景}
\vspace{-4pt}
\resumeSubheading
  {中南大学}{长沙, 湖南}
  {信息安全专业学士在读}{2019.9 - 2023.6}
  \resumeItemListStart
    \resumeItem{课程}
      {线性代数,概率与统计,计算机网络,编译器原理,操作系统与安全,
数据库原理、分布式系统与云计算、网络安全等。}
  \resumeItemListEnd

\vspace{-4pt}
%-----------Professional Experience-----------------
\section{专业经历}
  \resumeSubHeadingListStart
      
    \resumeSubheading
      {华为 - 2012 实验室}{西安}
      {软件开发工程师 @ 高斯实验室}{ 2022.7 - 现在}
      \resumeItemListStart
      	\resumeItem{内核裁剪}
		{数据库内核的裁剪和开发,通过将数据库运行内存占用率裁剪到原来的 10\%,解决了特殊情况下内存不足的问题。}
		\resumeItem{调研分享}
        {西安基地高斯部唯一同时向四个项目负责人做了项目演示和主题演讲的软件开发工程师实习生。}
      \resumeItemListEnd
          
    \resumeSubheading
      {Open Source Promotion Plan - ShardingSphere}{远程}
      {学生开发者}{ 2022.7 - 现在}
      \resumeItemListStart
        \resumeItem{开发}
          {设计了一个测试引擎,从 MySQL 和 PostgreSQL 动态加载超过 100,000 个 SQL 案例,并在 1 分钟内完成 SQL 解析器的参数化测试,并在远程集成测试,在本地也可以进行手动测试。}
        \resumeItem{优化}
          {将测试覆盖率报告从 Travis 迁移到 GitHub Action,并将CI时间至少减少 3-5 分钟。}
      \resumeItemListEnd
    
    \resumeSubheading
      {Cloud Native Computing Foundation - Glossary}{远程}
      {中文社区的联合创始人 \& 项目 Approver}{2022.2 - 现在}
      \resumeItemListStart
      	\resumeItem{开发 \& 维护}
		{负责 Glossary 的中文版本发布,截至目前已更新 50 个词汇。代表中文社区在 4 月份的 Glossary 工作组会议上发言。}
		\resumeItem{协作}
         {与其他创始人密切协调,进行 Glossary 工作组的建设。}
      \resumeItemListEnd

      
  \resumeSubHeadingListEnd
  
\vspace{-8pt}
%-----------Projects-----------------
\section{项目经历}
\resumeSubHeadingListStart 
   
         
   \resumeSubheadingCompact{CloudWeGo - Netpoll}{ 2022.7 - 2022.9}{远程开发}
   \resumeItemListStart
        \resumeItemSingle{设计了一个纯 Golang 版本 (没有 CGO) 的高性能 io\_uring I/O poller 的 SDK。}
        \resumeItemSingle{与 epoll 相比,大约改进了 3 倍,使用 io\_uring 实现了轮询器的默认 POLL 模式。}
    \resumeItemListEnd
\vspace{-4pt}
   \resumeSubheadingCompact{RoboCup China Open 2D Soccer Simulation}{2021.5 -  2021.9}
   {云麓校队队长, 中南大学}
      \resumeItemListStart
        \resumeItemSingle{结合模糊控制与强化学习,通过简化数学模型的设计,使控制机制和策略得到优化,优化后的射门命中率提升了21.36\%。}
        \resumeItemSingle{带领 CSU\_YUNLU 于 RoboCup China Open 2D Soccer Simulation 获得国家二等奖,并以第一作者于 ICRAIC 2021 发表论文。}
      \resumeItemListEnd

  \resumeSubHeadingListEnd
 
\vspace{-8pt}
%-----------Publication & Patents-----------------
\section{论文发表 \& 书籍出版}

    \resumeItem{}{\textbf{J., Yu}, Q., Zhao and W., Zhuang et al., "Decision and Evaluation of Ordering and Transshipment Schemes Based on Multi-objective Programming," 2021 5th Annual International Conference on Data Science and Business Analytics (ICDSBA), 2021, pp. 474-478, doi: 10.1109/ICDSBA53075.2021.00097.}
    
    \resumeItem{}{\textbf{Junfeng Yu} et al., "The Research of RoboCup2D Player Shooting Technique Based on Fuzzy Control," International Conference on Robotics Automation and Intelligent Control (ICRAIC 2021), 2022 J. Phys.: Conf. Ser. 2203 012059.}
    
    \resumeItem{}{\textbf{J., Yu}, (2022). \textit{Go 鲜为人知的角落}. Beijing: 图灵出版社(审核中)}
    
\vspace{-5pt}
%--------CONFERENCES & SERVICES------------
\section{活动 \& 奖项}
    
        \resumeItem{奖项}{获得国家级奖项2项(RoboCup中国2D足球模拟公开赛二等奖、2021年大学生创新创业大赛优秀结题),省级奖项3项(第十四届 ``挑战杯"大学生竞赛二等奖、第七届 ``互联网+"建行杯大学生竞赛二等奖、第十五届 ``升华杯 "大学生竞赛特等奖),校级奖项若干等。}
        \resumeItem{组织 \& 活动}{苹果实验室会长负责技术指导;计算机系年级长负责年级事务;代表学校协办了Ubuntu Kylin开源沙龙和线下发布会。}
        
\vspace{-4pt}
%-----------STATS-----------------
\section{专业技能}
  	\resumeItem
      {编程语言}
      {Golang(40k LOC), Java(20k LOC), C/C++(20k LOC), JavaScript, Python, ANTLR4}
    \resumeItem
      {框架 \& 技术栈}
      {数据库内核开发 (C/Java/ANTLR4), 基础架构开发 (Golang), Web 开发 (Java/Golang)}


%-------------------------------------------
\end{document}

func permute(nums []int) [][]int {
	res := make([][]int, 0)
	var dfs func(in, out []int)
	dfs = func(in, out []int) {
		if len(in) == 0 {
			res = append(res, out)
			return
		}
		for idx := range in {
			add := append(out, in[idx])
			left := in[:idx]
			if idx < len(in)-1 {
				left = append(left, in[idx+1:]...)
			}
			dfs(left, add)
		}
	}
	dfs(nums, []int{})
	return res
}
