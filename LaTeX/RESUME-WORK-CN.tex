\documentclass[letterpaper,10pt]{article}

\usepackage{latexsym}
\usepackage[empty]{fullpage}
\usepackage{titlesec}
\usepackage{marvosym}
\usepackage[usenames,dvipsnames]{color}
\usepackage{verbatim}
\usepackage{enumitem}
\usepackage[hidelinks]{hyperref}
\usepackage{fancyhdr}
\usepackage[english]{babel}
\usepackage[UTF8]{ctex}

\pagestyle{fancy}
\fancyhf{} % clear all header and footer fields
\fancyfoot{}
\renewcommand{\headrulewidth}{0pt}
\renewcommand{\footrulewidth}{0pt}

% Adjust margins
\addtolength{\oddsidemargin}{-0.6in}
\addtolength{\evensidemargin}{-0.5in}
\addtolength{\textwidth}{1.2in}
\addtolength{\topmargin}{-.6in}
\addtolength{\textheight}{1.0in}

\urlstyle{same}

\raggedbottom
\raggedright
\setlength{\tabcolsep}{0in}

% Sections formatting
\titleformat{\section}{
  \vspace{-8pt}\scshape\raggedright\large
}{}{0em}{}[\color{black}\titlerule \vspace{-6pt}]

%-------------------------
% Custom commands
\newcommand{\resumeItemSingle}[1]{
  \item\small{
    {#1 \vspace{-2pt}}
  }
}
\newcommand{\resumeItem}[2]{
  \item\small{
    \textbf{\:#1}{ #2 \vspace{-4pt}}
  }
}

\newcommand{\resumeSubheadingMininal}[1]{
  \vspace{-0pt}\item
    \begin{tabular*}{0.97\textwidth}[t]{l@{\extracolsep{\fill}}r}
      \textbf{#1}
    \end{tabular*}\vspace{-6pt}
}

\newcommand{\resumeSubheadingCompact}[2]{
\vspace{-0pt}\item
    \begin{tabular*}{0.97\textwidth}[t]{l@{\extracolsep{\fill}}r}
      \textbf{#1} & \textit{\small #2}
    \end{tabular*}\vspace{-1pt}
    \vspace{-5pt}
}

\newcommand{\resumeSubheadingCompactVertical}[2]{
	\vspace{-0pt}\item
    \begin{tabular*}{0.97\textwidth}[t]{l@{\extracolsep{\fill}}r}
      \textbf{#1}
    \end{tabular*}\vspace{-0pt}
    #2
    \vspace{-6pt}
}

\newcommand{\resumeSubheadingExtended}[3]{
  \vspace{-0pt}\item
    \begin{tabular*}{0.97\textwidth}[t]{l@{\extracolsep{\fill}}r}
      \textbf{#1} & #2
    \end{tabular*}\vspace{-0pt}
    #3
    \vspace{-6pt}
}

\newcommand{\resumeSubheadingWithTime}[3]{
  \vspace{-0pt}\item
    \begin{tabular*}{0.97\textwidth}[t]{l@{\extracolsep{\fill}}r}
      \textbf{#1} & \textit{\small #2} \\
    \end{tabular*}\vspace{-0pt}
    #3
    \vspace{-6pt}
}

\newcommand{\resumeSubheading}[4]{
  \vspace{-1pt}\item
    \begin{tabular*}{0.97\textwidth}[t]{l@{\extracolsep{\fill}}r}
      \textbf{#1} & #2 \\
      \textit{\small#3} & \textit{\small #4} \\
    \end{tabular*}\vspace{-6pt}
}

\newcommand{\resumeSubItem}[2]{
    \vspace{-1pt}\item
    \begin{tabular*}{0.97\textwidth}[t]{l@{\extracolsep{\fill}}r}
        \textbf{#1} & \textit{\samll #2}
    \end{tabular*}\vspace{-14pt}
    }

\renewcommand{\labelitemii}{$\circ$}

\newcommand{\resumeSubHeadingListStart}{\begin{itemize}[leftmargin=*]\vspace{-1pt}}
\newcommand{\resumeSubHeadingListEnd}{\end{itemize}\vspace{-2pt}}
\newcommand{\resumeItemListStart}{\begin{itemize}}
\newcommand{\resumeItemListEnd}{\end{itemize}\vspace{-5pt}}


%-------------------------------------------
%%%%%%  CV STARTS HERE  %%%%%%%%%%%%%%%%%%%%%%%%%%%%


\begin{document}

%----------HEADING-----------------
\begin{tabular*}{\textwidth}{l@{\extracolsep{\fill}}r}
  \textbf{\Large{余俊锋}} \\
  \centerline{
  \footnotesize \textbf{Portf:} \href{https://jacob953.com/}{https://jacob953.com/} |
  \textbf{Email:} \href{mailto:i@jacob953.com}{i@jacob953.com} | 
  \textbf{LinkedIn:} \href{https://www.linkedin.cn/incareer/in/junfeng-yu-578456248}{https://www.linkedin.com/in/jacob953/} |
  \textbf{Mobile:} +86 155 8521 0953}
\end{tabular*}
% \vspace{-10pt}
% \vspace{-4pt}
%-----------Education Background-----------------
% \vspace{-1pt}
\section{教育背景}
% \vspace{-4pt}
\resumeSubheading
  {中南大学}{长沙, 湖南}
  {信息安全学士在读}{2019.9 - 2023.6}
  \resumeItemListStart
    \resumeItem{课程}
      {线性代数,概率与统计,计算机网络,编译器原理,操作系统与安全,
数据库原理、分布式系统与云计算、网络安全等。}
  \resumeItemListEnd

% \vspace{-4pt}
%-----------Professional Experience-----------------
\section{专业经历}
  \resumeSubHeadingListStart
      
    \resumeSubheading
      {腾讯 - NoSQL}{成都}
      {研发工程师 @ 向量数据库}{ 2023.11 - 现在}
      \resumeItemListStart
      	\resumeItem{0-1-5-10 质量项目}
		{针对 7 种异常场景进行主动发现,能做出 23 种根因诊断,累计发现 175w 次,并配合诊断系统收敛至 1.5w 次,有效收敛率高达 99.1\%,并多次先于客户排查出根因,主动联系进行优化与调整,强化用户对产品的信任,以及发现了一些框架内部的潜在问题,保障项目质量的稳步迭代。}
      	\resumeItem{白屏化运维系统}
		{极大缩短了值班人员运营数据和运维操作的时间成本和沟通成本,通过质量大盘和实例管理将全地域问题事件级别的界面可视化,及其对应事件的诊断路由详情,并统一了各个中间件平台的入口,极大降低了问题排查的时间成本和沟通成本,同时,方便进行数据下钻,将深层数据进行日推同步。}
      	\resumeItem{PDF 解析算法}
		{自研算法通过工程能力实现 PDF 文件解析,覆盖绝大部分解析场景,包括图文解析、超小标题识别、表格跨页合并、段落跨页合并和多栏解析等,并将性能提升至接近原来的 1.5 倍,在串行解析模式下,平均每页解析耗时不超过 200 ms,大幅度提升了数据清洗和用户召回效果。}
      	\resumeItem{数据管道}
		{设计数据管道整体架构,面向异构数据源,可以无痛扩展不同的数据库类型和数据存储类型,并且能够满足未来流批一体的形态;主动搭建项目启动框架(服务启动、HealthCheck、预启动等),开发各种通用组件(任务状态机、日志组件等),帮助其他团队成员聚焦于业务开发。}
      \resumeItemListEnd
      
    \resumeSubheading
      {华为 - 2012 实验室}{西安}
      {软件开发工程师 @ 高斯实验室}{ 2022.7 - 2022.11}
      \resumeItemListStart
      	\resumeItem{内核裁剪}
		{独立负责数据库内核的裁剪,预研小内存场景下的数据库内核,将数据库由分布式裁剪为单体式,裁剪和优化多个模块,将内存占用率裁剪至原来的 10\%(由 1.6 GB 裁剪至 200 MB)。}
		\resumeItem{调研分享}
        {唯一同时向四个项目负责人做了项目演示,和 CCEH 论文调研成果分享的组内实习生。}
      \resumeItemListEnd
      
  %   \resumeSubheading
  %     {Open Source Promotion Plan - ShardingSphere}{远程}
  %     {学生开发者}{ 2022.7 - 2022.10}
  %     \resumeItemListStart
  %       \resumeItem{测试引擎}
  %         {设计了一个 SQL 解析的动态测试引擎,从 MySQL 和 PostgreSQL 加载超过 110,000 个 SQL 测试样例,并在 1 分钟内完成 SQL 解析器的参数化测试,可以由 GitHub Action 远程测试,在本地也可以进行手动测试。}
  %       \resumeItem{优化}
  %         {将测试覆盖率报告从 Travis 迁移到 GitHub Action,使 CI 时间至少节约了 3-5 分钟。}
  %     \resumeItemListEnd
    
  %   \resumeSubheading
  %     {CloudWeGO - Netpoll}{远程}
  %     {杰出贡献者}{2022.7 - 2022.9}
  %     \resumeItemListStart
  %     	\resumeItem{设计 SDK }
		% {设计了一个纯 Golang 版本 (没有 CGO) 的高性能 io\_uring I/O poller 的 SDK,采用共享内存的方案,通过设计 Submit 、 Complete 队列和内存屏障,实现了异步 I/O。}
		% \resumeItem{性能}
  %        {与 epoll 相比,SDK 性能大约提升了 3 倍,并使用 io\_uring 实现了轮询器的默认 POLL 模式。}
  %     \resumeItemListEnd

      
  \resumeSubHeadingListEnd
  
% \vspace{-8pt}
%-----------Projects-----------------
\section{项目经历}
\resumeSubHeadingListStart 

    \resumeSubheading
      {Tencent - Behavior3Gen}{成都}
      {项目维护人}{ 2024.11 - 现在}
      \resumeItemListStart
      	\resumeItem{设计 \& 实现}
		{开发 Behavior3 低代码平台规则引擎的二次开发工具,通过生成 .b3 文件简化节点的定制化开发负担,可以在静态编译阶段对服务启动 Panic 进行预防,并对个性化节点进行管理。}
		\resumeItem{内部分享}
        {在内网 KM 平台发表文章《Behavior3Gen:友好安全的 Behavior3 框架二次开发生成工具》被收录,并登上首页推荐。}
      \resumeItemListEnd

    \resumeSubheading
      {Open Source Promotion Plan - ShardingSphere}{远程}
      {社区开发者}{ 2022.7 - 2022.10}
      \resumeItemListStart
        \resumeItem{测试引擎}
          {设计了一个 SQL 解析的动态测试引擎,从 MySQL 和 PostgreSQL 加载超过 110,000 个 SQL 测试样例,并在 1 分钟内完成 SQL 解析器的参数化测试,可以由 GitHub Action 远程测试,在本地也可以进行手动测试。}
        \resumeItem{优化}
          {将测试覆盖率报告从 Travis 迁移到 GitHub Action,使 CI 时间至少节约了 3-5 分钟。}
      \resumeItemListEnd
    
    \resumeSubheading
      {CloudWeGO - Netpoll}{远程}
      {杰出贡献者}{2022.7 - 2022.9}
      \resumeItemListStart
      	\resumeItem{设计 SDK }
		{设计了一个纯 Golang 版本 (没有 CGO) 的高性能 io\_uring I/O poller 的 SDK,采用共享内存的方案,通过设计 Submit 、 Complete 队列和内存屏障,实现了异步 I/O。}
		\resumeItem{性能}
         {与 epoll 相比,SDK 性能大约提升了 3 倍,并使用 io\_uring 实现了轮询器的默认 POLL 模式。}
      \resumeItemListEnd

%    \resumeSubheadingCompact{车辆重识别防御系统}{ 2020.11 - 2021.11}{学生研究助理,中南大学}
%    \resumeItemListStart
%         \resumeItemSingle{参与构建基于 Transformer 的车辆重识别防御系统的特征提取和相关测试,并通过 Python 独立设计和实现 GLOM 设想。}
%         \resumeItemSingle{每 2 周在小组里报告 2 篇论文,并对神经网络的可解释性进行了独立研究。}
%     \resumeItemListEnd
% \vspace{-4pt}

%    \resumeSubheadingCompact{RoboCup China Open 2D Soccer Simulation}{2021.5 -  2021.9}
%    {云麓校队队长, 中南大学}
%       \resumeItemListStart
%         \resumeItemSingle{结合模糊控制与强化学习,通过简化数学模型的设计,使控制机制和策略得到优化,优化后的射门命中率提升了21.36\%。}
%         \resumeItemSingle{带领 CSU\_YUNLU 于 RoboCup China Open 2D Soccer Simulation 获得国家二等奖,并以第一作者于 ICRAIC 2021 发表论文。}
%       \resumeItemListEnd

  \resumeSubHeadingListEnd
 
% \vspace{-8pt}
%-----------Publication & Patents-----------------
% \section{论文发表 \& 书籍出版}

%     \resumeItem{}{\textbf{J., Yu}, Q., Zhao and W., Zhuang et al., "Decision and Evaluation of Ordering and Transshipment Schemes Based on Multi-objective Programming," 2021 5th Annual International Conference on Data Science and Business Analytics (ICDSBA), 2021, pp. 474-478, doi: 10.1109/ICDSBA53075.2021.00097.}
    
%     \resumeItem{}{\textbf{Junfeng Yu} et al., "The Research of RoboCup2D Player Shooting Technique Based on Fuzzy Control," International Conference on Robotics Automation and Intelligent Control (ICRAIC 2021), 2022 J. Phys.: Conf. Ser. 2203 012059.}
    
%     \resumeItem{}{\textbf{J., Yu}, (2022). \textit{Go 鲜为人知的角落}. Beijing: 图灵出版社(审核中)}
    
% \vspace{-5pt}
%--------CONFERENCES & SERVICES------------
% \section{活动 \& 奖项}
    
%         \resumeItem{奖项}{获得国家级奖项2项(RoboCup中国2D足球模拟公开赛二等奖、2021年大学生创新创业大赛优秀结题),省级奖项3项(第十四届 ``挑战杯"大学生竞赛二等奖、第七届 ``互联网+"建行杯大学生竞赛二等奖、第十五届 ``升华杯 "大学生竞赛特等奖),校级奖项若干等。}
%         \resumeItem{组织 \& 活动}{苹果实验室会长负责 Tech Talk;计算机学院年级长负责年级事务;代表学校协办了Ubuntu Kylin 开源沙龙和线下发布会。}
        
% \vspace{-4pt}
%-----------STATS-----------------
\section{专业技能}
  	\resumeItem
      {编程语言}
      {Golang(40k LOC), Java(20k LOC), C/C++(20k LOC), JavaScript, Python, ANTLR4}
    \resumeItem
      {框架 \& 技术栈}
      {数据库内核开发 (C/Java/ANTLR4), 基础架构开发 (Golang), Web 开发 (Java/Golang)}


%-------------------------------------------
\end{document}
